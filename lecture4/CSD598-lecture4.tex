\documentclass[12pt,aspectratio=169,english,finnish,pdflatex%,handout
]{beamer}
\definecolor{MyGreen}{RGB}{50, 120, 50}
\usecolortheme[named=MyGreen]{structure}

\usepackage{babel}
\usepackage[utf8]{inputenc}
\usepackage[T1]{fontenc}
\usepackage{amsmath,amssymb} 
\usepackage{animate}
\usepackage{multimedia}

\usepackage{natbib}
\bibpunct[: ]{(}{)}{,}{}{}{;}

\usepackage{tikz}

\usepackage{tipa}

\usepackage{hyperref}

\setbeamertemplate{navigation symbols}{}

\graphicspath{{figures/}}

\setlength{\leftmargini}{0pt}
\setlength{\leftmarginii}{1em}

%% Write out the names of graphics files being included 
\newwrite\graphics
\immediate\openout\graphics=\jobname.graphics%
\let\oincludegraphics\includegraphics% store original \includegraphics
\renewcommand{\includegraphics}[2][]{% prepend to it (could also use xpatch, etc.)
  \immediate\write\graphics{#2}
  \oincludegraphics[#1]{#2}}

\newcommand\Wider[2][3em]{%
\makebox[\linewidth][c]{%
  \begin{minipage}{\dimexpr\textwidth+#1\relax}
  \raggedright#2
  \end{minipage}%
  }%
}

\newcommand{\kommentti}[1]{
  {\bf[#1]}
}


\title{CSD 598 - Winter 2026
  \\~\\
  Descriptive statistics and data visualisation
}
\author{Pertti Palo} 
\date{3 Feb 2026} 


\begin{document}

\frame{\titlepage
  \centering
} 

\frame{\frametitle{Last time: Ethics}
  Any questions? Any thoughts?
}

\frame{
  \centering
  {
    \bf \Large 
    \usebeamercolor[fg]{title}
    Why do data exploration?
    \vfill
    \includegraphics[height=.8\textheight]{messing-around-vs-science.png}
    {\tiny No Boilerplate: Hack your Brain with Powerlifting \url{https://www.youtube.com/watch?v=WxN9VzwoGGw}} 
  }
}


\frame{
  \centering
  {
    \bf \Large 
    \usebeamercolor[fg]{title}
    Descriptive stats    
    \vfill
  }
}


\frame{\frametitle{Central tendencies}
\begin{columns}
  \begin{column}{.5\textwidth}
    \includegraphics[height=.9\textheight]{Visualisation_mode_median_mean.png}
  \end{column}

  \begin{column}{.5\textwidth}
    Cmglee, CC BY-SA 3.0, via Wikimedia Commons
  \end{column}
\end{columns}
}


\frame{\frametitle{Dispersion and shape}

\begin{columns}
  \begin{column}{.35\textwidth}
    Dispersion
    \begin{itemize}
      \item Range, quartiles: easy to calculate on samples
      \item Standard deviation (SD), variance: most used
      \item Average absolute deviation: more robust than SD
    \end{itemize}
    \vspace{.5cm}
    Shape measures
    \begin{itemize}
      \item Skewness: which way does the distribution lean?
      \item Kurtosis: tailedness, things get complex here
    \end{itemize}
  \end{column}

  \begin{column}{.65\textwidth}
    \includegraphics[width=\columnwidth]{Relationship_between_mean_and_median_under_different_skewness.png}

    Diva Jain, CC BY-SA 4.0 international, via Wikimedia Commons
  \end{column}
\end{columns}
}


\frame{\frametitle{Dependence between variables: The usual}
  \begin{itemize}
    \item Covariance: measure of the linear relationship between two variables.
    \item Pearson's $\rho$: correlation of continuous variables
    \item Spearman's $\rho$: correlation of ordering of variables
  \end{itemize}
  \vspace{.5cm}
  \includegraphics[width=.3\columnwidth]{Spearman_fig1.svg.png}
  \includegraphics[width=.3\columnwidth]{Spearman_fig2.svg.png}
  \includegraphics[width=.3\columnwidth]{Spearman_fig3.svg.png}

  Skbkekas, CC BY-SA 3.0, via Wikimedia Commons
}


\frame{\frametitle{Dependence between variables: Pearson vs Distance correlation}
  \begin{columns}
    \begin{column}{.49\textwidth}
      \includegraphics[width=\columnwidth]{Correlation_examples2.svg.png}
      Pearson correlation
    \end{column}
    \begin{column}{.49\textwidth}
      \includegraphics[width=\columnwidth]{Distance_Correlation_Examples.svg.png}
      Distance correlation
    \end{column}
  \end{columns}
  \vspace{.5cm}
  Denis Boigelot, CC BY-SA 3.0, via Wikimedia Commons
}


\frame{
  \centering
  {
    \bf \Large 
    \usebeamercolor[fg]{title}
    Visualisation
    
    \vfill

    \includegraphics[width=.4\textwidth]{not_a_pie_chart.jpg}
  }
}


\frame{\frametitle{Choosing a good graph}

}


    Tell-show-do-apply


\frame{\frametitle{References}
  
\bibliographystyle{apalike}
\bibliography{science_combined}

}
\end{document}

