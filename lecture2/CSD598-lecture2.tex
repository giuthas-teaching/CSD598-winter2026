\documentclass[12pt,a4,english,finnish,pdflatex%,handout
]{beamer}
\definecolor{MyGreen}{RGB}{50, 120, 50}
\usecolortheme[named=MyGreen]{structure}

\usepackage{babel}
\usepackage[utf8]{inputenc}
\usepackage[T1]{fontenc}
\usepackage{amsmath,amssymb} 
\usepackage{animate}
\usepackage{multimedia}

\usepackage{natbib}
\bibpunct[: ]{(}{)}{,}{}{}{;}

\usepackage{tikz}

\usepackage{tipa}

\usepackage{hyperref}

\setbeamertemplate{navigation symbols}{}

\graphicspath{{figures/}}

\setlength{\leftmargini}{0pt}
\setlength{\leftmarginii}{1em}

\newcommand\Wider[2][3em]{%
\makebox[\linewidth][c]{%
  \begin{minipage}{\dimexpr\textwidth+#1\relax}
  \raggedright#2
  \end{minipage}%
  }%
}

%% Write out the names of graphics files being included 
\newwrite\graphics
\immediate\openout\graphics=\jobname.graphics%
\let\oincludegraphics\includegraphics% store original \includegraphics
\renewcommand{\includegraphics}[2][]{% prepend to it (could also use xpatch, etc.)
  \immediate\write\graphics{#2}
  \oincludegraphics[#1]{#2}}

\newcommand{\kommentti}[1]{
  {\bf[#1]}
}


\title{CSD 598 - Winter 2026
  \\~\\
  % Epistemology and Ontology
  % \\\&\\
  Some thoughts on\\Silent Knowledge in Doing Science
}
\author{Pertti Palo} 
\date{20 Jan 2026} 


\begin{document}

\frame{\titlepage
  \centering
} 


% \frame{
%   \centering
%   {
%     \bf \Large 
%     \usebeamercolor[fg]{title}
%     Epistemology and Ontology
    
%     \vfill
% %    \includegraphics[height=1.5cm]{figures/aalto_logo} 
%   }
% }


% \frame{\frametitle{Definitions - one take}
% \Wider{
%   \begin{description}[\leftmargin=20pt]
%     \item[Epistemology] "is the branch of philosophy that examines the nature,
%     origin, and limits of knowledge. [\ldots] Epistemologists study the
%     concepts of belief, truth, and justification to understand the nature of
%     knowledge. To discover how knowledge arises, they investigate sources of
%     justification, such as perception, introspection, memory, reason, and
%     testimony."
%     \item[Ontology] "is the philosophical study of being. [\ldots] To articulate
%     the basic structure of being, ontology examines the commonalities among all
%     things and investigates their classification into basic types, such as the
%     categories of particulars and universals. [\ldots] Systems of categories aim
%     to provide a comprehensive inventory of reality by employing categories such
%     as substance, property, relation, state of affairs, and event."
%   \end{description}
%   See \cite{Wikipedia-Epistemology-2025,Wikipedia-Ontology-2026}
%   }
% }


% \frame{\frametitle{Questions and thoughts to keep you thinking}
% \begin{itemize}
%   \item What do we assume to be true when using statistics?
%   \item What do we assume to be true when doing empirical research?
%   \item Does (can) one experiment refute generations of knowledge?
%   \item Science is (also) an oral tradition. 
%   \item Science relies heavily on (somewhat formalised) storytelling.
% \end{itemize}
% }


% \frame{
%   \centering
%   {
%     \bf \Large 
%     \usebeamercolor[fg]{title}
%     Discussion about the practical side of doing science
    
%     \vfill
% %    \includegraphics[height=1.5cm]{figures/aalto_logo} 
%   }
% }


\frame{\frametitle{If we are pressed for time}
Possible topics are:
\begin{itemize}
  \item Reading
  \item Note-taking in various contexts
  \item Tools of note-taking
  \item Writing
  \item Blank page fixes
  \item Dealing with problems
\end{itemize}
\vspace{.5cm}
We will definitely cover
\begin{itemize}
  \item Power differentials
  \item Relating to ideas
  \item Pacing
\end{itemize}
}


\frame{\frametitle{Reading}
\begin{itemize}
  \item How to read a paper? What's your strategy?
  \item A lot of articles and books and resources are available through U of A.
  % \item The process of doing research is not linear, unlike the stories in
  % literature would have you believe.
\end{itemize}
}


\frame{\frametitle{What to record when doing research}
\begin{itemize}
  \item This applies to studying, reading (also known as studying), recording
  data, analysing data, and writing up the results.
  \item Reflect on what you read: It's a good idea to keep a record of your
  developing understanding and especially of the questions you have.
  \item It's also a good idea to keep a record of any anomalies (or lack
  thereof) in data gathering because that makes analysis easier.
\end{itemize}
}


\frame{\frametitle{Tools of note-taking}
\begin{itemize}
  \item Notebooks are good and there are way more possibilities out there
  beyond a regular composition notebook, which is a very solid choice though.
  \item Knowledge management systems such as Obsidian, Notion, Logseq and
  others can also be very useful, but might be more difficult and distracting
  to use during data recording.
  is here. 
  \item If you like formal systems look up Zettelkasten.
  \item Reflect on what you read: Keep your note-taking tools handy because
  idle minds produce excellent ideas.
\end{itemize}
}


\frame{\frametitle{Getting into writing}
\begin{itemize}
  \item Authorship guidelines and who gets invited to contribute to a paper.
  \item The process is not linear, unlike the stories in literature would have
  you believe.
  \item Falsifiability and its problems.
  \item How to separate personal experience from publishable stories while
  being honest.
  \item Rationalisation of results - Don't try to publish only negative results
  unless you see papers like that already done.
  \item Talk to people. Learn to scale the amount of detail you give.
\end{itemize}
}


\frame{\frametitle{Dealing with blank page: Quick fixes}
\begin{itemize}
  \item One of the easiest solutions is: Make a mistake and then correct it.
  \item If you foresee this cropping up, you can fix it before you get there by
  note-taking: 
  \begin{itemize}
    \item Reflect on other peoples texts. Even by just putting the article info
    on the top of a page and then writing about the article as you read it
    underneath.
    \item Reflect on your own work. Guess what? You can apply the same
    principle to planning research, to recording data, to analysing data.
  \end{itemize} 
\end{itemize}
}


\frame{\frametitle{Dealing with blank page: Slow fix}
\begin{itemize}
  \item If you have a lot of time -- a month or two -- to prepare, take up
  journaling. 
  \begin{itemize}
    \item It does not really matter what you write about as long as you do so regularly. 
    \item It will improve your writing in general.
    \item It will most likely make the fear of blank page disappear.
    \item And journaling has other benefits.
  \end{itemize}
  \item And seek people who are willing to read your text and give it positive,
  constructive feedback.
  \item Negative feedback leads to avoidance behaviours.
\end{itemize}
}


\frame{\frametitle{How to deal with problems}
\begin{itemize}
  \item Problems need about three (short) attempts before contacting the boss.
  \begin{itemize}
    \item This is not because you are expected to solve problems on your own.
    \item It's because those documented attempts can provide a more experienced
    person with information they need for solving the problem.
  \end{itemize}
  \item Don't try to solve things for too long on your own either. 
  \begin{itemize}
    \item If you can't solve it, get help.
    \item Also, if you don't know how to attempt a solution, get help.
  \end{itemize}
  \item If something comes up that will change plans -- things take longer,
  you can't work as fast as you thought, etc. -- let others know earlier
  rather than later and they'll like you for it.
\end{itemize}
}


\frame{\frametitle{Power differentials}
\begin{itemize}
  \item Academia is littered with power differentials.
\end{itemize}
}


\frame{\frametitle{Recycle ideas, throw away ideas}
\begin{itemize}
  \item Don't reinvent the wheel unless absolutely necessary.
  \item Don't get married to your ideas.
  \item Intuition takes time to work.
  \item We all need good working environments.
\end{itemize}
}


\frame{\frametitle{Pace yourself, celebrate often}
\begin{itemize}
  \item It will take longer than you think.
  \item It actually still does with 25 years of experience.
  \item Get enough rest.
  \item While they can be impressive, sprints can't last much longer than a
  week.
  \item Our work is ephemeral. Take the time to mark and celebrate each
  milestone.  
\end{itemize}
}


\frame{\frametitle{References}

Many thanks to the CSB Monday morning speech coffee group for essentially
writing most of this lecture.

% \bibliographystyle{apalike}
% \bibliography{science_combined}

}
\end{document}

