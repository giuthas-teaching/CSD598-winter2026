\documentclass[12pt,aspectratio=169,english,finnish,pdflatex%,handout
]{beamer}
\definecolor{MyGreen}{RGB}{50, 120, 50}
\usecolortheme[named=MyGreen]{structure}

\usepackage{babel}
\usepackage[utf8]{inputenc}
\usepackage[T1]{fontenc}
\usepackage{amsmath,amssymb} 
\usepackage{animate}
\usepackage{multimedia}

\usepackage{natbib}
\bibpunct[: ]{(}{)}{,}{}{}{;}

\usepackage{tikz}

\usepackage{tipa}

\usepackage{hyperref}

\setbeamertemplate{navigation symbols}{}

\graphicspath{{figures/}}

\setlength{\leftmargini}{0pt}
\setlength{\leftmarginii}{1em}

%% Write out the names of graphics files being included 
\newwrite\graphics
\immediate\openout\graphics=\jobname.graphics%
\let\oincludegraphics\includegraphics% store original \includegraphics
\renewcommand{\includegraphics}[2][]{% prepend to it (could also use xpatch, etc.)
  \immediate\write\graphics{#2}
  \oincludegraphics[#1]{#2}}

\newcommand\Wider[2][3em]{%
\makebox[\linewidth][c]{%
  \begin{minipage}{\dimexpr\textwidth+#1\relax}
  \raggedright#2
  \end{minipage}%
  }%
}

\newcommand{\kommentti}[1]{
  {\bf[#1]}
}


\title{CSD 598 - Winter 2026
  \\~\\
  Metalevel look at CSD 501
}
\author{Pertti Palo} 
\date{10 Feb 2026} 


\begin{document}

\frame{\titlepage
  \centering
} 

\frame{
  \centering
  {
    \bf \Large 
    \usebeamercolor[fg]{title}
    Preliminaries: Epistemology and Ontology

    Tell-show-do-apply    \vfill
%    \includegraphics[height=1.5cm]{figures/aalto_logo} 
  }
}


\frame{\frametitle{Definitions}
\Wider{
  \begin{description}[\leftmargin=20pt]
    \item[Epistemology] "is the branch of philosophy that examines \ldots"
    \item[Ontology] "is the philosophical study of being\ldots"
  \end{description}
  See \cite{Wikipedia-Epistemology-2025,Wikipedia-Ontology-2026}
  }
}

\frame{\frametitle{Causality}
\begin{itemize}
  \item We all know - or should know - by now that correlation does not imply
  causation.
  \item But what is causality?
  \begin{itemize}
    \item David Hume has said that causality can not be proven empirically
    because we can not know if the world changes at a fundamental level.
    \item Immanuel Kant has replied to this that causality is not a feature of
    the universe, but rather a way or tool of reasoning.
    \item In addition, some phenomena in quantum physics break causality
    empirically. And in the general theory of relativity we can not say what
    the order of two events occurring is because it depends on the point of
    observation.
    \item And this could all be wrong, but it's unlikely (or seems unlikely
    given the evidence).
  \end{itemize}
  \item Unless this was not news, it should make one think. And that's what I'd
  like to encourage in any case.
\end{itemize}
}

\frame{\frametitle{Causality and other tools}
\begin{itemize}
  \item It is not enough to know how to use a tool, we should also understand
  enough of its workings that we know
  \begin{itemize}
    \item the limits of our understanding
    \item the limits of the tool
  \end{itemize}
  \item Statistics is a tool. Like causality.
\end{itemize}
}


\frame{\frametitle{Reminder: Questions and thoughts to keep you thinking}
\begin{itemize}
  \item What do we assume to be true when using statistics?
  \item What do we assume to be true when doing empirical research?
  \item Does (can) one experiment refute generations of knowledge?
  \item Science is (also) an oral tradition. 
  \item Science relies heavily on (somewhat formalised) storytelling.
\end{itemize}
}


\frame{
  \centering
  {
    \bf \Large 
    \usebeamercolor[fg]{title}
    Descriptive statistics and data visualisation
    
    \vfill
%    \includegraphics[height=1.5cm]{figures/aalto_logo} 
  }
}



\frame{\frametitle{References}
  
\bibliographystyle{apalike}
\bibliography{science_combined}

}
\end{document}

